\documentclass[UTF8]{ctexbook}

\usepackage{graphicx}
\usepackage{tikz}
\usetikzlibrary{positioning}

\title{\heiti \Large 计算机科学基础}
\author{陈相安}
\date{\today}  
\begin{document}
    \pagenumbering{arabic}
    \maketitle
    \newpage
    \tikz[remember picture, overlay] 
    \node at (current page.center){\Large 致所有我爱和爱我的人};
    \tableofcontents
    \chapter{绪论}
        \section{序言}
        相比较语言的学习,CSE的基础课程同样的重要。我时常称这些看起来并没有什么用处的课程
        称为内功心法,而语言的学习更像是外家功夫。就好比金庸先生笔下的郭靖一样,倘若他没有
        浑厚的内力作为支撑,想必降龙十八掌的威力对他来说也要削弱不少。因此在这个计算机行业内卷的
        时代,学习这些基础课程是相当必要的。

        本书是在我自己学习过程中进行的一份总结和笔记。其中囊括了绝大部分计算机科学的基础课程,
        相比于专业性强的书籍,这更像是一本指南和教程,在阐述原理的同时加入了一些实践性质的内容。
        旨在给自己的知识做出总结及分享给同样需要学习CSE相关专业或者有兴趣学习的伙伴们一个学习的
        知识框架体系和资料。
        
        \newpage
        \section{前言}
            这是一本面向中文读者的计算机科学的指导工具书,为了使各种程度的读者都能相对简单的学习,
            本书尽可能少的设计相关的语言和长篇大论的理论知识。诚然,不清楚理论知识是不行的,对于
            部分数学逻辑和相关知识理论似乎也是不可避免的。因此,本书的读者需要对计算机科学和数学有
            一些了解,更适合与本科一年级以上的读者以及具有相关类似背景的读者进行阅读。
            
            全书共分为八章,大体上可分为三部分,第一部分包括第2-3章,主要介绍了程序代码的实现与底层
            逻辑,重点体现在数据的存储方式和解决算法上;第二部分包括第4-6章,是计算机的构成和运行方式
            的一个内容;第三部分包括第7-8章,主要是拓展计算机科学的内容,阐明了离散变量对计算机发展的
            作用和原理,同时加入了编译原理,有助于了解计算机时如何执行解释代码段。
            
            本书除绪论外,每一章的每一小节均有5-10题不等的习题及1-3题不等的实践内容,习题主要用于巩固
            读者对该节内容理论方面的理解,实践则是为了提升读者的运用能力和思考能力。带有*号的题目则相对
            来说更加有挑战性,需要读者串联更多的知识而不单单仅是本节的内容。

            本书不可避免的会涉及大量外国人名、专业术语等,倘若简单全部翻译成中文,则读者在日后阅读更深层次
            的国外论文、书籍时容易产生陌生感,不利于进一步的学习,因此,本书仅对部分外语进行了翻译,其他则保持
            英语。当然,在附录中你可以找到相关对应的中文名称。同时,对于偏向数学的学科,也会涉及到大量的数学
            符号和公式,限于篇幅,亦不作一一解释,均在附录中会进行解释和翻译。

            计算机科学近些年发展极其迅速,目前以及时当下几乎最火爆的行业,学科涉及也越来越广袤,作为一个普通
            的CSE学科的学习者,笔者自认为才疏学浅,仅略知CSE的皮毛而已,更兼时间、精力和学识所限,书中谬误
            之处难以避免,若蒙读者诸君不吝告知,余不胜感激。

            陈相安

            2020年11月
    \chapter{数据结构}
        \section{线性表}
        线性表是简单、基本、常用的数据结构。线性表是线性结构的抽象 (Abstract),线性结构的特点是结构中的数据元素之间存在一对一的线性关系。
        这种一对一的关系指的是数据元素之间的位置关系,即:

        除第一个位置的数据元素外,其它数据元素位置的前面都只有一个数据元素;
        除后一个位置的数据元素外,其它数据元素位置的后面都只有一个元素。也就是说,数据元素是 一个接一个的排列。因此,可以把线性表想象为一种数据元素序列的数据结构。
            \subsection{顺序表}
            \subsection{链表}
        \section{串}
        \section{堆}
        \section{栈}
        \section{队列}
        \section{哈希表}
        \section{树}
        \section{图}
        \section{特殊数据结构}
            \subsection{线段树}
            \subsection{并查集}
            \subsection{字典树}
    \newpage
    \chapter{算法思想}
        \section{算法概论}
        \section{递归算法}
        \section{分治算法}
        \section{双指针算法}
        \section{动态规划算法}
            \subsection{贪心算法}
        \section{回溯算法}
        \section{深度优先与广度优先算法}
        \section{排序算法}
        \section{查找算法}
        \section{字符串相关算法}
    \newpage
    \chapter{计算机操作系统}
        \section{操作系统概论}
        \section{进程管理}
        \section{内存管理}
        \section{文件管理}
        \section{I/O管理}
        \section{操作系统安全性}

    \newpage
    \chapter{计算机网络}
        \section{网络结构绪论}
        \section{物理层}
        \section{数据链路层}
        \section{网络层}
        \section{传输层}
        \section{会话层}
        \section{表示层}
        \section{应用层}
        \section{网络安全与管理}
    \newpage
    \chapter{计算机组成原理}
        \section{绪论}
        \section{语法制导简述}
        \section{词法分析}
        \section{语法分析}
        \section{语法制导翻译}
        \section{中间代码的生成}
        \section{运行时环境}
        \section{代码生成}
        \section{机器无关优化}
        \section{指令级并行}
        \section{并行与局部优化}
        \section{过程间分析}
    \newpage
    \chapter{编译原理}
        \section{绪论}
        \section{数据表示和运算}
        \section{存储器层次结构}
        \section{指令系统}
        \section{CPU运行流程}
        \section{总线}
        \section{输入输出系统}
    \newpage
    \chapter{计算机数学拓展-离散数学基础}
        \section{绪论}
        \section{逻辑与证明}
        \section{基本结构}
        \section{算法分析}
        \section{数论与密码学}
        \section{归纳与递归}
        \section{计数}
        \section{离散概率}
        \section{关系}
        \section{图论}
        \section{树}
        \section{计算模型}
    \newpage
    \chapter{附录A:数学符号意义}
    \chapter{推荐读物}
    \chapter{参考文献}
    \chapter{练习题答案}
    \chapter{后记}
\end{document}

